
\documentclass{article}

\usepackage[english]{babel}
\usepackage[square,numbers]{natbib}
\bibliographystyle{plainnat}
\setcitestyle{authoryear}

\usepackage{daniels}
% Includes
%   amsmath
%   amssymb
%   amsthm
%   latexsym
%   color
%   mathtools
%   dsfont
%   xcolor with dvipsnames
%   braket
%   accents
%   musicography
%   stmaryrd
%   enumerate 
%   algorithm2e
%   mdframed
%   listings

% ==============================
%           Details   
% ==============================

\title{\LaTeX~Generic Template}
\author{Max Daniels\\ \texttt{daniels.g@northeastern.edu}} 
\date{Northeastern University --- \today}

%----------------------------------------------------------------------------------------

\begin{document}

\maketitle 

% ==============================
%           Introduction   
% ==============================

\section*{Introduction}
This is a general purpose \LaTeX~style file for notes and assignments. Some specifications on the numbering:
\begin{enumerate}
    \item Available environments for theorems are: \texttt{theorem}, \texttt{prop}, \texttt{cor}, \texttt{lemma}, \texttt{dfn}
    \item These environments \textit{share global ordering}, scoped inside each individual section. For example, Section 1 might contain Theorem 1.1, Lemma 1.2, Proposition 1.3 in order. 
    \item There are additional environments available for homework assignments: \texttt{exr}, \texttt{claim}
    \item Claims are scoped inside of individual exercises. Exercises have global scope and share an ordering.
\end{enumerate}
There is also support for math typesetting:
\begin{theorem}[Residue Theorem]
Let $f: \C \to \C$ be a complex function which is meromorphic on the interior of a closed curve $C$. Let $\{z_i\}_{i=1}^n$ be the set of singularities of $f$. Then,
\begin{equation*}
    \oint_{C} f \, dx = \frac{1}{2\pi i} \sum_{i=1}^n \text{Res}(z_i)
\end{equation*}
\end{theorem}
\begin{proof}
This is a proof environment.
\end{proof}
\begin{rem}
The residue theorem is often used with Cauchy's integral formula to reduce a contour integral to a summation over factors of a function evaluated at its singularities. 
\end{rem}

\begin{exr}[Given]
This is an exercise.
    \begin{claim}
    This is a claim
    \end{claim}
\end{exr}

\section{Citations}
Citations are in author-year format with Natbib, like so: \citep{lamport1994latex}.
\section{Numbering Demonstration}
Aliquam arcu turpis, ultrices sed luctus ac, vehicula id metus. Morbi eu feugiat velit, et tempus augue. Proin ac mattis tortor. Donec tincidunt, ante rhoncus luctus semper, arcu lorem lobortis justo, nec convallis ante quam quis lectus. Aenean tincidunt sodales massa, et hendrerit tellus mattis ac. Sed non pretium nibh.

\begin{theorem}
Theorem theorem theorem.
\end{theorem}
\begin{cor}
Corollary. 
\end{cor}

%----------------------------------------------------------------------------------------
%	PROBLEM 1
%----------------------------------------------------------------------------------------

\section{Other Nice Features} % Numbered section

In hac habitasse platea dictumst. Curabitur mattis elit sit amet justo luctus vestibulum. In hac habitasse platea dictumst. Pellentesque lobortis justo enim, a condimentum massa tempor eu. Ut quis nulla a quam pretium eleifend nec eu nisl. Nam cursus porttitor eros, sed luctus ligula convallis quis. Nam convallis, ligula in auctor euismod, ligula mauris fringilla tellus, et egestas mauris odio eget diam. Praesent sodales in ipsum eu dictum.

\begin{question}[\itshape (with optional title)]
	This is a question box.
	\begin{enumerate}[(a)]
		\item Do this.
		\item Do that.
		\item Do something else.
	\end{enumerate}
\end{question}
	
\begin{center}
	\begin{minipage}{0.5\linewidth}
		\begin{algorithm}[H]
			\KwIn{$(a, b)$, two floating-point numbers}  
			\KwResult{$(c, d)$, such that $a+b = c + d$} 
			\medskip
			\If{$\vert b\vert > \vert a\vert$}{
				exchange $a$ and $b$ \;
			}
			$c \leftarrow a + b$ \;
			$z \leftarrow c - a$ \;
			$d \leftarrow b - z$ \;
			{\bf return} $(c,d)$ \;
			\caption{\texttt{FastTwoSum}} 
			\label{alg:fastTwoSum}  
		\end{algorithm}
	\end{minipage}
\end{center}

\begin{file}[hello.py]
\begin{lstlisting}[language=Python]
#! /usr/bin/python

import sys
sys.stdout.write("Hello World!\n")
\end{lstlisting}
\end{file}

% Command-line "screenshot"
\begin{commandline}
	\begin{verbatim}
		$ chmod +x hello.py
		$ ./hello.py

		Hello World!
	\end{verbatim}
\end{commandline}

\newpage
\bibliography{citations}
\end{document}
